\documentclass{article}
\usepackage[utf8]{inputenc}

\title{Avaliação de conhecimento 2 \\ INF05508 - Lógica de computação \\
        Instituto de Informática - UFRGS
    }
\author{Luís Eduardo Pereira Mendes}
\date{Abril 2022}

\usepackage{natbib}
\usepackage{graphicx}


\usepackage[utf8]{inputenc}
\usepackage[lmargin=3cm,tmargin=3cm,bmargin=3cm,rmargin=3cm]{geometry}
\usepackage[brazil]{babel}
\usepackage[T1]{fontenc}
\usepackage[onehalfspacing]{setspace}
\usepackage{logicproof}
\everymath{\displaystyle}

\usepackage{amsmath,amsthm,amssymb,amsfonts,dsfont}
\newcommand{\RR}{\mathds{R}}
\newcommand{\CC}{\mathds{C}}
\newcommand{\NN}{\mathds{N}}
\newcommand{\ZZ}{\mathds{Z}}
\newcommand{\QQ}{\mathds{Q}}

\begin{document}

\maketitle
\begin{enumerate}
    \item Prove usando as regras de inferência de Dedução Natural, para a Lógica de Predicados e justifique a generalização da conclusão:

    
    \begin{enumerate}
    
    
        \item $\forall x (P(x) \to C(x)), \forall x (C(x) \to V(x) ) \vdash \forall x(P(x) \to V(x))$ 
            \setlength\subproofhorizspace{2em}
            \begin{logicproof}{1}
                \forall x (P(x) \rightarrow C(x)) & Premissa\\
                \forall x (C(x) \rightarrow V(x)) & Premissa\\
                \begin{subproof}
                    \llap{$x_0\quad$} P(x_0) \rightarrow C(x_0) & $\forall x \mathrm{e} 1 $\\
                    
                    C(x_0) \rightarrow V(x_0) & $\forall x \mathrm{e} 1 $\\
                    p(x_0) \rightarrow V(x_0) & $S.H. 3,4$ 
                \end{subproof}
                \forall x (Q(x)) & $\forall x \mathrm{i} 3-5$
            \end{logicproof}
        \item $\forall x(F (x) \lor  G(x)) \vdash  \exists x(F (x) \lor  G(x))$
            \begin{logicproof}{0}
                \forall x (F(x) \lor G(x)) & premissa\\
                F(x_0) \lor G(x_0) & $\forall x \mathrm{e}, 1$\\
                \exists x (F(x) \lor G(x)) & $\exists x \mathrm{i}, 2$
            \end{logicproof}
        
        
        \end{enumerate}
        
    \item Justifique a seguinte prova na linha 4 da demonstração
    $$(\forall x)(P(x) \to  Q(x)), (\forall y)P(y)\vdash  (\forall x)(Q(x))$$

    \setlength\subproofhorizspace{2cm}
        \begin{logicproof}{1}
            \forall x (P(x) \to Q(x)) & premissa \\
            \forall y (P(y)) & premissa \\
            \begin{subproof}
                \llap{$x_0\quad$} P(x_0) \to Q(x_0) & $\forall x \mathrm{e}, 1 $ \\
                P(x_0) & $\forall x \mathrm{e}, 2 $ \\
                Q(x_0) & $\to \mathrm{e}, 3,4 $
            \end{subproof}
            \forall x (Q(x)) & $\forall x \mathrm{i}, 3-5 $
        \end{logicproof}
        Na linha 3 é feita uma particularização para uma nova variável.
        Na linha 4 é possível usar a mesma variável que já foi usada anteriormente na prova já que é uma particularização do universal, e, como é intuitivo, se vale para qualquer x, então podemos escolher qualquer valor para x.
        Na linha 5 é aplicado uma exclusão da implicação (modus ponens)
        Na linha 6 é feita uma generalização do universal, que é possível pois foi provado que é válido para um $x_0$ qualquer, então, consequentemente, é válido para qualquer x.
    \item A ordem dos quantificadores. Diga quais permutações podem ser executadas sem mudar o significado das formulas:
    \begin{enumerate}
        \item "$\exists x\forall y (L(x,y))$" para "$\forall y\exists x (L(x,y))$" 
        \item "$\exists x\exists y (L(x,y))$" para "$\exists y\exists x (L(x,y))$" 
    \end{enumerate}

    Apenas no item b não faz diferença a ordem dos quantificadores, uma vez que, quando os quantificadores são iguais, não importa a ordem que eles estejam.
    
    \item $\phi$
    
    \item Faça a transliteração da seguinte sentença:
        Nem todo filósofo inglês é esquisito\\
        Use:
        $F(x)$ filósofo,
        $I(x)$ inglês,
        $E(x)$ esquisito 

    $\exists x ((F(x) \land I(x)) \to \neg E(x)) $
    \item Diga se as fórmulas a seguir são válidas, inválidas ou satisfazíveis e inválidas
        \begin{enumerate}
            \item $\forall xP(x) \to  \exists xP(x)$  \\
                Válida
            \item $\forall xP(x) \to  P(a) $\\
                Válida
            \item $\exists x \forall y (p(y) \to  q(x, y))$\\
                Satisfazível
        \end{enumerate}
        
        
    \item Tendo-se por base as regras de generalização do existencial diga quais generalizações podem ser realizadas a partir das premissas.
    
    Parece fazer sentido que se uma propriedade vale para uma coisa particular, então a propriedade vale para alguma coisa. Por exemplo, sabemos que 5 é um número primo, e faz sentido concluir que existe algum número primo. Seja P(x) = “x é um numero primo”, então podemos inferir $\exists x P(x)$ de $P(5)$. Se W é uma wff podemos inferir $\exists x W (x)$ de $W(c)$ para uma constante $c$? 
Podemos inferir $\exists x W(x)$ de $W(x)$? 
Podemos inferir $\exists x W(x)$ de $W(t)$ para qualquer termo $t$? 
Na seguinte prova, assuma que estamos falando sobre os números naturais: 
\begin{logicproof}{0}
    \forall x\exists y (x<y) & P\\
    \exists y (x<y) & 1 UE\\
    x<c & 2 EE\\
    \exists x (x<x) & 3 EI \\
    \exists z (x < z) & 3 EI\\ 
    \exists x (x < c) & 3 EI
\end{logicproof}

Para inferir $\exists x W(x)$ de $W(x)$ temos que mostrar que é válida para algum valor x.
Para inferir $\exists x W(x) de Wt$ para um termo t qualquer, deve valer que: $Wt = W(x)[x/t]$. 
Par

Não é possível realizar a generalização da linha 4, pois a expressão não é válida para nenhum caso, já que claramente nenhum x é maior do que ele mesmo.\\
A generalização da linha 5 é válida pois uma variável z qualquer é maior do que algum x.\\
A generalização existencial da linha 6 é possível, pois foi particularizado na linha 3 que x < c, então, existe x tal que x < c.





\end{enumerate}
\end{document}
